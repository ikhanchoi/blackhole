\documentclass[noamsfonts,a4paper,10pt]{amsart}
\usepackage[margin=2cm]{geometry}
\usepackage[bitstream-charter,cal]{mathdesign}
%\usepackage{hyperref}
\usepackage{setspace}
\onehalfspacing
\renewcommand{\arraystretch}{0.8}

\theoremstyle{plain}
\newtheorem{theorem}{Theorem}[section]
\newtheorem*{theorem*}{Theorem}
\newtheorem{lemma}[theorem]{Lemma}
\newtheorem*{lemma*}{Lemma}
\newtheorem{corollary}[theorem]{Corollary}
\newtheorem*{corollary*}{Corollary}
\theoremstyle{definition}
\newtheorem{definition}[theorem]{Definition}
\newtheorem*{definition*}{Definition}


\title{Black hole simlations}

\begin{document}
\maketitle


\section{Einstein field equation}




\section{Schwarzschild metric}


A \emph{pseudo-group} on a topological space is a wide subgroupoid of the groupoid of open subsets satisfying a sheaf property.

(or inverse semi-groups?)


Let $U$ be a Einstein Lorentzian four-dimensional manifold.
Suppose a local action of $\mathrm{SO}(3)\times\mathbb{R}$ on $U$ is given.
We assume that the metric is invariant under this local action.
This is the definition of Schwarzschild metrics.
There only one parameter, the mass of the black hole....

\begin{definition}[Schwarzschild metric]
spherically symmetric static vacuum solution
\end{definition}

coordinate functions naturally arising from a continuous Lie group action?


The line element of the Schwarzschild metric is given in the Schwarzschild coordinates as
\[ds^2=-\left(1-\frac{r_S}r\right)c^2\,dt^2+\left(1-\frac{r_S}r\right)^{-1}dr+r^2\,d\Omega,\]
where
\[r_S:=\frac{2GM}{c^2},\qquad d\Omega=d\theta^2+\sin^2\theta\,d\varphi^2\]


\subsection{Geodesic motions}

for observers, for rays


\section{Kerr metric}

\begin{definition}[Kerr metric]
\end{definition}


\subsection{Boyer-Lindquist coordinates}

For $G=c=1$

$M$, $J$, $Q$

$Q$ Carter constant, charge?

\[\]


\subsection{Singularities}



\section{Geodesic equation}


\subsection{Constant of motions}


\subsection{Initial condition transformation}

How can we set a hyperplane $M$ with an observer in the Boyer-Lindquist coordinates?

How can we transform the ray-casting direction in the Boyer-Lindquist coordinates.

\end{document}

